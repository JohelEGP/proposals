%!TEX root = std.tex


\rSec0[paper.design]{Design}

\rSec1[paper.design.overview]{Overview}

\pnum
We hope that this library encourages the development of well-specified generic components that operate on numbers.
\ref{paper.design} summarizes what is precisely defined in \ref{paper.api}.
Additionally, we discuss what might not be evident in that reference documentation.

\rSec1[paper.design.traits]{Type traits}

\pnum
The type traits are mostly opt-in.
They are:
\begin{itemize}
\item The \cname{number} opt-ins, \tcode{enable_number} and \tcode{enable_complex_number}.
\item The identities, \tcode{number_zero} and \tcode{number_one}.
\item The associated types, \tcode{number_difference_t} and \tcode{vector_scalar_t}.
\end{itemize}

\pnum
They are based on existing practice.
That is evident for the first two groups in \ref{paper.api}.
Like \tcode{std::iter_reference_t} and \tcode{std::ranges::range_value_t},
the associated types are named \tcode{\placeholder{kind-of-structure}_\placeholder{associated-type}_t}.

\rSec1[paper.design.concepts]{Concepts}

\rSec2[paper.design.concepts.utils]{Utilities}

\pnum
\cname{number} is the least constrained number concept.
\cname{common_number_with} is used to relate similar types.
\begin{codeblock}
template<typename T>
concept number = enable_number_v<T> && std::regular<T>;

template<typename T, typename U>
concept common_number_with =
  number<T> && number<U> && std::common_with<T, U> && number<std::common_type_t<T, U>>;
\end{codeblock}

\rSec2[paper.design.concepts.order]{Ordering}

\pnum
\cname{ordered_number} requires a model of \tcode{std::\cname{totally_ordered}}.

\pnum
\cname{number_line} is the concept that requires the \hrefVIII{complete set} of increment and decrement.
It is named after the term described in \hrefIX{the Wikipedia article ``Number line''}.
\begin{note}
\ecname{decrementable}\irefcppx{range.iota}{concept:decrementable} doesn't work for number types.
\begin{itemize}
\item
It's for integer-like types\irefcpp{iterator.concept.winc}.
Or we'd have to specialize \tcode{std::incrementable_traits}\irefcpp{incrementable.traits}
with a \tcode{difference_type} that is an integer-like type
and that is different to \tcode{number_difference_t}\iref{num.assoc.types}.
\item
It requires \tcode{std::\cname{regular}}.
Feedback at \hrefVII{} is that it over-constrains expression templates that can't be default-initialized.
\end{itemize}
\end{note}

\rSec2[paper.design.concepts.arithmetic]{Arithmetic}

\pnum
Then follow the building blocks of the form \cname{\opt{compound_}\placeholder{operation}_with}.

\pnum
\ecname{addition-with} requires that \tcode{+} performs an addition\irefiev{102-01-11}.
But addition is associative, i.e., $a + (b + c) = (a + b) + c$.
The feedback from \refp{2402} and \hrefVI{} is that
\begin{itemize}
\item associativity is over-constraining, and
\item users generally accept ``associative enough''.
\end{itemize}

\pnum
It occurred to us that we can work around this issue in the same way integer type division\irefcppx{expr.mul}{4.sentence-3} works.
\tcode{3 / 2} is conceptually carried out in real numbers, resulting in $1.5$.
Then, the operator \tcode{/} yields $1.5$ with the fractional part discarded, resulting in \tcode{1}.

\pnum
\ecname{addition-with} is specified similarly to integer type division.
That relaxes the associativity to ``associative enough''.
\ecname{addition-with} requires that \tcode{a + b} performs the addition $F$ in an unspecified set.
Then, it requires that \tcode{+} yields $F$ by mapping it to a value of the type of the result.

\pnum
First, we introduce an unspecified set.
The over-constraining associative addition is done in this unspecified set.
Then, we introduce a mapping from the addition to the result of \tcode{+}.
The mapping is left unspecified, so the result isn't over-constrained.

\pnum
The operations required by other refinements of \cname{number} are specified similarly.

\rSec2[paper.design.concepts.terse]{Terse syntax}

\pnum
Some concepts ending in \tcode{_for} just require a single concept ending in \tcode{_with} with inverted arguments.
\begin{codeblock}
template<typename T, typename U>
concept modulus_for = modulo_with<U, T>;
template<typename T, typename U>
concept vector_space_for = point_space_for<U, T>;
\end{codeblock}
These enable the terse concept syntax.
\begin{example}
\begin{codeblock}
template<vector_space_for<Number> Number2> auto& operator+=(vector<Number2>);
auto operator%(modulus_for<Number> auto);
\end{codeblock}
\end{example}

\rSec2[paper.design.concepts.algebraic]{Algebraic structures}

\pnum
Finally, the building blocks are used to define some helpers.
These helpers are used to define heterogeneous concepts on algebraic structures.
There also are homogeneous counterparts with a default.

\begin{simpletypetable}
{Examples of models of algebraic structure concepts}
{paper.design.concepts.algebraic}
{ll}
\topline
\lhdr{Concept}            & \rhdr{Close model}                              \\ \capsep
\cname{point_space}       & \tcode{std::chrono::sys_seconds}                \\
\cname{f_vector_space}    & \tcode{std::chrono::seconds}                    \\
\cname{field_number}      & \tcode{std::complex<double>}                    \\
\cname{field_number_line} & \tcode{double}                                  \\
\cname{scalar_number}     & \tcode{double} or \tcode{std::complex<double>}  \\
\end{simpletypetable}
