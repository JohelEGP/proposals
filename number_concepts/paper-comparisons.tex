%!TEX root = std.tex


\rSec0[paper.cmp]{Similar works}

\rSec1[paper.p1813]{P1813}

\pnum
The algorithms from \tcode{<numeric>} still don't have constrained counterparts in \Cpp{}23's \tcode{std::ranges}.
\refp{1813} aimed to remedy that.
Its approach to specifying algebraic structures was bottom-up.
The feedback \refp{2402} and \hrefVI{} gave was that this is over-constraining.

\pnum
This paper's approach is instead top-down.
Having started from Bjarne's \tcode{Number}\iref{paper.history},
we have refactored the existing concepts as needs arise.

\pnum
\tcode{<numeric>} algorithms work with numeric expressions or have overloads with function objects.
The precise suitability of our number concepts for constrained overloads hasn't been studied.
The unconstrained versions \tcode{std::accumulate} already allows accumulating on a \tcode{std::string}.
We would expect its constrained version to allow the same.
