%!TEX root = std.tex


\rSec0[paper.history]{History}

\pnum
\hrefI{After finding the pixel}, Johel wrote generic components on numbers.
But those components failed to abide to the \Cpp{} Core Guidelines'
\begin{itemize}
\item \hrefII{I.9: If an interface is a template, document its parameters using concepts} and
\item \hrefIII{T.concepts: Concept rules}.
\end{itemize}

\pnum
So Johel armed himself with GCC's Concepts TS support,
and eventually the LLVM fork with \Cpp{} standard concepts.
He took the \tcode{Number} concept from \hrefIV{Bjarne's presentation} as a starting point.
His main concerns then were to
\begin{itemize}
\item specify the semantics and
\item to refactor it to support \tcode{std::chrono}'s \tcode{duration} and \tcode{time_point}.
\end{itemize}

\pnum
He stumbled upon \refcppx{intro.defs}{2.2} while looking for an answer to these concerns.
The vocabulary of the first subject area of the Electropedia,
\hrefV{Mathematics - General concepts and linear algebra},
would serve as building blocks to solve these concerns.

\pnum
From there, Johel reviewed the feedback on \refp{1813} by the slides \refp{2402} and its proposal \hrefVI{}.
\ref{paper.design} and \ref{paper.evo} explain the actions taken and thoughts had based on these reviews.

\pnum
The library then remained vitally unchanged for years.
Johel had hardly advanced his own application in the meantime.
That is what had spurred the inception and growth of the library.

\pnum
Finally, on 2023-09-21, this library was proposed for merging into the mp-units library~(\hrefVII{}).
This spurred a revitalization on the improvement of the design of the library.
