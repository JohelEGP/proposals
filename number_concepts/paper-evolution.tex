%!TEX root = std.tex


\rSec0[paper.evo]{Evolution}

\rSec1[paper.evo.relax]{Relax to generalize}

\pnum
The concepts are known to be over-constraining for templates like \hrefX{\tcode{boost::hana::int_c}}.
The feedback from \hrefVII{} suggests its time for the hierarchy to be refactored.
This will allow a wider variety of number templates that make interesting uses of the type system.
\wheredevelisat

\rSec1[paper.evo.ops]{More operations}

\pnum
There are operations beyond those required by the definition of a vector space\irefiev{102-03-01}.

\pnum
\refpx{2980R0}{systems-of-quantities} needs some of those to specify some quantities.
Just like there are scalar quantities\irefiev{102-02-19}, there are vector and tensor quantities.
An example is speed\irefiev{113-01-33} (a scalar quantity),
defined as the magnitude\irefiev{102-03-23} of the velocity\irefiev{113-01-32} (a vector quantity).
Here are other operations from \hrefXI{}:
\begin{codeblock}
template<typename T>
concept VectorRepresentation = /* ... */ &&
  requires(T a, T b, /* ... */) {
    /* ... */
    { dot_product(a, b) } -> Scalar;
    { cross_product(a, b) } -> Vector;
    { tensor_product(a, b) } -> Tensor;
    { norm(a) } -> Scalar;
  };
template<typename T>
concept TensorRepresentation = /* ... */ &&
  requires(T a, T b, /* ... */) {
    /* ... */
    { tensor_product(a, b) } -> Tensor;
    { inner_product(a, b) } -> Tensor;
    { scalar_product(a, b) } -> Scalar;
  };
\end{codeblock}

\pnum
Scalars, vectors, and tensors are vector spaces\irefiev{102-03-01}.
The number concepts library has \cname{scalar_number}\iref{num.concepts}, which represents a scalar\irefiev{102-02-18}.
\cname{vector_space} could be refined with these additional operations to support vectors and tensors.

\pnum
But vector is an overloaded word.
\hrefXII{Its Wikipedia article} says
\begin{quote}
In mathematics and physics,
vector is a term that refers colloquially to some quantities
that cannot be expressed by a single number (a scalar),
or to elements of some vector spaces.
\end{quote}
\cname{vector_space} can be renamed to its alternative term \cname{linear_space}.
Then, we could add \cname{vector} for the colloquial "tuple of 2 or more scalars" and \cname{tensor} as refinements.

\pnum
The \Cpp{} standard library offers many more operations on arithmetic types,
which are models of \cname{scalar_number}.
\hrefXIII{mp-units also offers a subset of those for \tcode{mp_units::quantity}},
whose specializations model \cname{vector_space}.
The current design of this number concepts library doesn't consider generalizing this.

\pnum
\wheredevelisat

\rSec1[paper.evo.mixed.expr]{Mixed expressions}

\pnum
A \term{mixed expression} is a \hrefXIV{mathematical expression} with more than one operation.
\hrefVI{} describes,
in particular in \href{https://wg21.link/P1673R12#generalizing-associativity-helps-little}{10.8.4},
why concepts generally are of little help.

\pnum
The number concepts only require expressions with up to two operands.
But an implementation of an algorithm might need to use mixed expressions.
The unfeasible alternative would be to include all possible constraints an algorithm might use.

\pnum
For example, the magnitude of a 2-dimensional Cartesian vector is $|\vec{v}| = \sqrt{x^2 + y^2}$\irefiev{102-03-23}.
Consider a \Cpp{} representation like this:
\begin{codeblock}
template<scalar_number Number>
struct vector2d {
  Number x;
  Number y;

  // A faster alternative when just ordering by magnitude.
  auto square_magnitude() { return x * x + y * y; }
};
\end{codeblock}
\cname{scalar_number} requires that the type of \tcode{Number} squared models \tcode{\cname{common_number_with}<Number>}.
But it doesn't require that adding squared \tcode{Number}s works.
The best we could do is to add the additional constraint for the result of \tcode{Number} squared.
It could be \cname{point_space}, the weakest constraint, as required by the algorithm.
Or it could be \cname{scalar_number}, the same constraint as \tcode{Number}, as users would expect of the return type.
In code, it would look like a trailing \tcode{requires \cname{scalar_number}<decltype(x * x)>}.
That doesn't scale to mixed expressions with more operations.

\pnum
One of the points \hrefVI{} summarizes is:
\begin{quote}
\begin{itemize}
\item
We can and do use existing Standard language, like \tcode{\placeholder{GENERALIZED_SUM}},
for expressing permissions that algorithms have.
\end{itemize}
\end{quote}
For \tcode{\placeholder{GENERALIZED_SUM}}, see \refcpp{numeric.ops}.
We agree with \hrefVI{} on this point.
But that doesn't preclude constraining algorithms with number concepts.

\pnum
There is still value in \hrefII{I.9} and \hrefIII{T.concepts}.
The number concepts in this library
already don't require the result of addition to be associative\iref{paper.design.concepts.arithmetic}.
It's also possible to require that a mixed expression works as one would naturally expect.
This can be done by adding an extra semantic requirement on refinements of \cname{number}.

\pnum
How to deal with this lies in the use of \cname{common_number_with} in refinements of \cname{number}.
\tcode{\cname{common_number_with}<T, U>} subsumes \tcode{std::\cname{common_with}<T, U>}\irefcpp{concept.common}.
Although it's not part of \tcode{std::\cname{common_with}},
\tcode{std::common_type_t<T, U>} also generally behaves like \tcode{T} and \tcode{U}.

\pnum
This is the idea for the semantic requirement on refinements of \cname{number}.
If \tcode{\cname{common_number_with}<T, U>} is required,
the user can use \tcode{T} in place of \tcode{U}, and
the semantic constraints on \tcode{U} also apply to \tcode{T}.

\pnum
With this requirement in mind, let's reconsider this example:
\begin{codeblock}
auto square_magnitude() { return x * x + y * y; }
\end{codeblock}
\tcode{\cname{scalar_number}<Number>} requires:
\begin{codeblock}
{ c * u } -> common_number_with<V>;
\end{codeblock}
That's essentially:
\begin{codeblock}
{ x * x } -> common_number_with<Number>;
\end{codeblock}
Even if the type of \tcode{Number} squared isn't the same as \tcode{Number},
the semantic constraints of \tcode{\cname{scalar_number}<decltype(x * x)>} apply
when the user writes one of its required expressions.
In this case, the algorithm is required to work as expected.

\pnum
It is the intent that this doesn't preclude the \tcode{+} on squared \tcode{Number}s from being ill-formed.
It is also the intent that the \tcode{+} on squared \tcode{Number}s
is still not required to be a total function\irefcppx{structure.requirements}{8}
or to be valid for all input values\irefcppx{concepts.equality}{2}.
I.e., the result can still be incorrect or exit via an exception.

\pnum
The worth of this formulation is that you can meet \hrefII{I.9} and \hrefIII{T.concepts}.
Even if an algorithm still needs to be more specific, as with \tcode{\placeholder{GENERALIZED_SUM}}.

\pnum
\wheredevelisat
