%!TEX root = std.tex


\rSec1[scope]{Scope}

\pnum
\indextext{scope|(}%
This document describes the contents of the \defn{mp-units library}.
\indextext{scope|)}


\rSec1[refs]{References}

\pnum
\indextext{references|(}%
The following documents are referred to in the text
in such a way that some or all of their content
constitutes requirements of this document.
For dated references, only the edition cited applies.
For undated references,
the latest edition of the referenced document
(including any amendments) applies.
\begin{itemize}
\item
The \Cpp{} Standards Committee.
N4892: \doccite{Working Draft, Standard for Programming Language \Cpp{}}.
Edited by Thomas K\"{o}ppe.
Available from: \url{https://wg21.link/N4892}
\item
The \Cpp{} Standards Committee.
P1841R1: \doccite{Wording for Individually Specializable Numeric Traits}.
Edited by Walter E. Brown.
Available from: \url{https://wg21.link/P1841R1}
\item
The \Cpp{} Standards Committee.
SD-8: \doccite{Standard Library Compatibility}.
Edited by Bryce Lelbach.
Available from: \url{https://wg21.link/SD8}
\item
ISO 80000-2:2019, \doccite{Quantities and units --- Part 2: Mathematics}
\item
IEC 60050 (all parts),
\doccite{International Electrotechnical Vocabulary (IEV)}
\item
IEC 60050-102:2007/AMD3:2021,
\doccite{Amendment 3 --- International Electrotechnical Vocabulary (IEV) ---
Part 102: Mathematics --- General concepts and linear algebra}
\item
IEC 60050-112:2010/AMD2:2020,
\doccite{Amendment 2 --- International Electrotechnical Vocabulary (IEV) ---
Part 112: Quantities and units}
\end{itemize}

\pnum
N4892 is hereinafter called \defn{\Cpp{}}.

\pnum
IEC 60050 is hereinafter called \defn{IEV}.
\indextext{references|)}


\rSec1[defs]{Terms and definitions}

\pnum
\indextext{definitions|(}%
For the purposes of this document,
the terms and definitions given in
\Cpp{},
IEC 60050-102:2007/AMD3:2021,
IEC 60050-112:2010/AMD2:2020,
the terms, definitions, and symbols given in
ISO 80000-2:2019,
and the following apply.

\pnum
ISO and IEC maintain terminology databases
for use in standardization
at the following addresses:
\begin{itemize}
\item ISO Online browsing platform: available at \url{https://www.iso.org/obp}
\item IEC Electropedia: available at \url{http://www.electropedia.org}
\end{itemize}

\pnum
Terms that are used only in a small portion of this document
are defined where they are used and italicized where they are defined.

\indexdefn{modulo}%
\definition{modulo}{def.mod}
operation performed on a set for which a division\irefiev{102-01-21} and an addition\irefiev{102-01-11} are defined,
the result of which, for elements $a$ and $b$ of the set,
is the unique element $r$, if it exists in the set,
such that $a = \lfloor a/b \rfloor b + r$


\rSec1[spec]{Specification}

\rSec2[spec.ext]{External}

\pnum
The specification of the mp-units library subsumes
\refcpp{description}, \refcpp{requirements}, \refcpp{concepts.equality}, and SD-8,
all assumingly amended for the context of this library.
\begin{note}
This means that, non exhaustively,
\begin{itemize}
\item \tcode{::mp_units2} is a reserved namespace, and
\item
\tcode{std::vector<mp_units::\placeholdernc{type}>}
is a program-defined specialization and a library-defined specialization
from the point of view of \Cpp{} and this library, respectively.
\end{itemize}
\end{note}

\pnum
The mp-units library is not part of the \Cpp{} implementation.

\rSec2[spec.cats]{Categories}

\pnum
Detailed specifications for each of the components in the library are in
\ref{\firstlibchapter}--\ref{\lastlibchapter},
as shown in \tref{lib.cats}.

\begin{floattable}{Library categories}{lib.cats}
{ll}
\topline
\hdstyle{Clause}        & \hdstyle{Category}                              \\ \capsep
\ref{nums}              & Numbers library                                 \\
\end{floattable}

\pnum
The numbers library\iref{nums}
describes components for dealing with numbers.

\rSec2[headers]{Headers}

\pnum
The mp-units library provides the
\defnx{mp-units library headers}{header!mp-units library},
shown in \tref{headers.mp.units}.

\begin{multicolfloattable}{mp-units library headers}{headers.mp.units}
{l}
\tcode{<mp-units/numbers.h>} \\
\end{multicolfloattable}

\rSec2[spec.reqs]{Library-wide requirements}

\rSec3[spec.res.names]{Reserved names}

\pnum
The mp-units library reserves macro names that start with
\tcode{MP_UNITS\opt{\gterm{digit-sequence}}_}.
